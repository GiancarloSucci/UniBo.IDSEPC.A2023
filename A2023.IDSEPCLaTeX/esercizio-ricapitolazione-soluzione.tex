\documentclass[12pt]{article}

\usepackage{url}

\title{Dieci segreti per fare una bella presentazione scientifica}
\author{Il lettore}

\begin{document}
\maketitle

\section{Introduzione}

Il testo di questo esercizio \`{e} una versione notevolmente ridotta e leggermente modificata dell'ottimo articolo omonimo (in inglese) di Mark Schoeberl e Brian Toon:
\url{http://www.cgd.ucar.edu/cms/agu/scientific_talk.html} e poi tradotta e modificata dal docente del corso.

\section{I segreti}

Ho compilato questo elenco personale di ``Segreti'' ascoltando presentazioni efficaci e terribili. Non pretendo che questo elenco sia completo: sono sicuro che ci sono cose che ho tralasciato. Ma la mia lista probabilmente copre circa il 90\% del da farsi.

\begin{enumerate}

\item Prepara il tuo materiale in modo accurato e logico. Strutturalo come una storia. Non ci sono scuse per la mancanza di preparazione.

\item Esercitati e ripeti la tua presentazione, magari davanti a uno specchio.

\item Non inserire troppo materiale. Pochi punti ben detti passano molte pi\`{u} informazioni di un minestrone.

\item Evita le formule ogni qual volta puoi. Si dice che per ogni formula nella tua presentazione, il numero di persone ti seguono si dimezza. Quindi, se $q$ \`{e} il numero di formule nella presentazione, $n$ \`{e} il numero di persone presenti e $m$ quelle che capiscono, abbiamo che:
\begin{equation}
m =   \frac{n}{2^q}
\end{equation}

\item La conclusione deve essere di pochi punti: non si riescono a ricordare pi\`{u} di un paio di concetti da una presentazione.

\item Parla al pubblico e non allo schermo.

\item Evita di produrre suoni fastidiosi. Cerca di evitare ``Ummm'' o ``Ahhh'' tra le frasi.

\item Ripulisci per bene la tua struttura grafica. Ecco un elenco di suggerimenti per una grafica migliore:

\begin{itemize}
\item Usa lettere grandi.

\item Mantieni la grafica semplice. Non mostrare ci\`{o} che non ti serve.

\item Usa il colore.

\end{itemize}

\item Sii affabile ed empatico nel rispondere alle domande.

\item Usa l'umorismo ogni qual volta possibile.

\end{enumerate}

\end{document}
